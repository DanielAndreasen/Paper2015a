%
\documentclass{aa}
% \documentclass[referee]{aa}
\usepackage[varg]{txfonts}
\usepackage[separate-uncertainty=true]{siunitx}
\usepackage[version=3]{mhchem}

\sisetup{range-units = brackets}

\def\eps{\varepsilon}
\def\aap{A\&A}
\def\eprint{e-prints}
\def\apj{ApJ}
\def\apjs{ApJS}
\def\apjl{ApJL}
\def\mnras{MNRAS}
\def\aj{AJ}
\def\nat{Nature}
\def\aaps{A\&A Supp.}
\def\prd{Phys. Rev. D}
\def\prl{Phys. Rev. Lett.}
\def\araa{ARA\&A}       % Annual Review of Astron and Astrophys

\begin{document}


\title{Near-IR spectroscopy of the Sun and HD20010}
\subtitle{Compiling a new line list in the near-IR}


\author{ D.~T.~Andreasen\inst{1,2}
    \and S.~G.~Sousa\inst{1}
    \and E.~Delgado Mena\inst{1}
    \and N.~C.~Santos\inst{1,2}
    \and M.~Tsantaki\inst{1}
    \and B.~Rojas-Ayala\inst{1}
    \and V.~Neves\inst{3}}


\institute{
Instituto de Astrof\'isica e Ci\^encias do Espa\c{c}o, Universidade do Porto, CAUP, Rua das Estrelas, 4150-762 Porto, Portugal
\email{daniel.andreasen@astro.up.pt}
\and
Departamento de F\'isica e Astronomia, Faculdade de Ci\^encias, Universidade do Porto, Rua Campo Alegre, 4169-007 Porto, Portugal
\and
Departamento de F\'{i}sica, Universidade Federal do Rio Grande do Norte, 59072-970 Natal, RN, Brazil
}





\date{Received ...; accepted ...}

\abstract
% Context
{Effective temperature, surface gravity, and metallicity are basic
spectroscopic stellar parameters necessary to characterize a star or a
planetary system. Reliable atmospheric parameters for FGK stars have
been obtained mostly from methods that rely on high resolution and high
signal-to-noise optical spectroscopy. The advent of a new generation
of high resolution near-IR (NIR) spectrographs opens the possibility
of using classic spectroscopic methods with high resolution and high
signal-to-noise in the NIR spectral window.}
% Aims
{We compile a new iron line list in the NIR from a solar spectrum to
derive precise stellar atmospheric parameters, comparable to the ones
already obtained from high resolution optical spectra. The spectral
range covers \SI{10000}{\angstrom} to \SI{25000}{\angstrom}, which is
equivalent to the Y, J, H, and K bands.}
% Methods
{Our spectroscopic analysis is based on the iron excitation and
ionization balance done in LTE. We use a high resolution and high
signal-to-noise ratio spectrum of the Sun from the Kitt Peak telescope
as a starting point to compile the iron line list. The oscillator
strengths ($\log\mathit{gf}$) of the iron lines were calibrated for the
Sun. The abundance analysis was done using the MOOG code after measuring
equivalent widths of 357 solar iron lines.}
% Results
{We successfully derived stellar atmospheric parameters for the Sun.
Furthermore, we analysed HD20010, a F8IV star, from which we derived
stellar atmospheric parameters using the same line list as for the Sun.
The spectrum was obtained from the CRIRES-POP database. The results
are compatible with the values found in the literature, confirming the
reliability of our line list. However, we obtain large errors due to the
quality of the data.}
% Conclusions
{}



\keywords{data reduction: high resolution spectra --
          stars individual: HD20010 --
          stars individual: Sun}
\maketitle



\section{Introduction}
\label{sec:introduction}

Effective temperature ($T_\mathrm{eff}$), surface gravity ($\log g$),
and metallicity ([M/H], where iron is normally used as a proxy)
are fundamental atmospheric parameters necessary to characterise a single
star, and to determine other indirect fundamental parameters
such as mass, radius, and age from stellar evolutionary models
\citep[see e.g.][]{Girardi2000,Dotter2008,Baraffe2015}.
Precise and accurate stellar parameters are also essential in
exoplanet searches. Planetary radius and mass are mainly found from
lightcurve analysis and radial velocity analysis, respectively. The
determination of the mass of the planet implies a knowledge of the
stellar mass, while the measurement of the radius of the planet
is dependent on our capability to derive the radius of the star
\citep[see e.g.][]{Torres2008,Ammler2009,Torres2012}.

The derivation of precise stellar atmospheric parameters is not a simple
task. Different approaches often lead to discrepant results \citep[see
e.g.][]{Santos13}. Interferometry is usually considered  an accurate
method for deriving stellar radii \citep[e.g.][]{Boyajian2012}; however,
it is only applicable for bright nearby stars. Asteroseismology, on the
other hand, reveals the inner stellar structure by observing the stellar
pulsations at the surface. From asteroseismology it is possible to
measure the surface gravity and mean density, and therefore to calculate
the mass and radius \citep[e.g.][]{Kjeldsen1995}.

A crucial parameter for the indirect determination of stellar bulk
properties is the effective temperature. In that respect, the infrared
flux method (IRFM) has proven to be reliable for FGK dwarf and
subgiant stars. However, the IRFM needs a priori knowledge of the
bolometric flux, reddening, surface gravity, and stellar metallicity
\citep{Blackwell1977,Ramirez2005b,Casagrande2010}.

Finally, the use of high resolution spectroscopy along with stellar
atmospheric models is an extensively tested method that allows
the derivation of the fundamental parameters of a star \citep[see
e.g.][]{Santos13,Valenti2005}. The procedure depends on the quality
of the spectra, their resolution, and wavelength region. For low
resolution spectra ($\lambda/\Delta\lambda < 20\,000$) the
preferred method is to fit the overall observed spectrum with a synthetic one
\citep[see e.g.][]{Recio2006}. Higher resolution spectra of
slowly rotating stars (below 10 to 15 \si{km/s})  are in the regime
where the equivalent width (EW) method can be used (for details see
Sect.~\ref{sec:method}).

The derivation of stellar atmospheric parameters from high
resolution spectra in the optical is now based on a standard
procedure \citep[see e.g.][]{Valenti2005,Sousa2008a}. With the
advancement of high resolution near-infrared (NIR) instruments, we will now be able to use a similar
technique to that used in the optical part of the spectrum
\citep[see e.g.][]{Melendez1999,Sousa2008a,Tsantaki2013,Mucciarelli2013,Bensby2014}.
At the moment, the GIANO spectrograph installed at \emph{Telescopio Nazionale Galileo} (TNG) is already
available \citep{GIANO}, as is the \emph{infrered doppler instrument} (IRD) installed
at the Subaru telescope \citep{IRD}. Three new spectrographs are planned for the
near future: 1) \emph{Calar Alto high-Resolution search for M dwarfs with Exoearths with
Near-infrared and optical Échelle Spectrographs} (CARMENES) for the \SI{3.5}{m} telescope at Calar Alto
Observatory \citep{CARMENES} had first light in December 2014,
2) The \emph{CRyogenic InfraRed Echelle Spectrograph Upgrade Project} (CRIRES+) at the
\emph{Very Large Telescope} (VLT) \citep{CRIRESp} with expected first light in 2017,
and 3) \emph{un SpectroPolarimètre Infra-Rouge
A Near-InfraRed Spectropolarimeter} (SPIRou) at \emph{The Canada-France-Hawaii Telescope} (CFHT) \citep{SPIROU1,SPIROU2} with expected first light
in 2017 as well. The spectral resolutions for these spectrographs range
between $50\,000$ and $100\,000$.

Even though reliable line lists for the derivation of stellar parameters
using optical spectra exist, the situation is different in the near-IR
regime. There are a few for the synthesis method (e.g. \citet{Onehag2012,Origlia2013,Rhodin2015}),
 and there is  the large general compilation by
\citet{Melendez1999}. Moreover, there are line lists compiled
in the NIR for the \emph{Apache Point Observatory Galactic Evolution Experiment} (APOGEE) survey \citep[see e.g.][]{Smith2013,Shetrone2015}.
Thus, in this paper we want to explore the
possibility of creating a line list of iron lines in the NIR which can
be applied for FGKM stars optimized for the EW method in a consistent
way, as is currently done for these stars in the optical (FGK
only). The paper is organized as follows. In Sect.~\ref{sec:method}
we present how to compile a line list and the method for deriving
parameters with the equivalent width method for an iron line list.
In Sect.~\ref{sec:results} we present the results for the derived
parameters for the Sun and HD20010. In
Sect.~\ref{sec:conclusion}  we discuss our results.


\section{Method}
\label{sec:method}

The two most widely used methods for deriving stellar atmosphere
parameters from a spectrum are spectral synthesis and the equivalent width (EW) method.
The spectral synthesis method compares synthetic spectra to an observed
spectrum and finds the best model by a minimization procedure
\citep[see e.g.][]{Valenti2005,Onehag2012,Blanco2014}. When the minimization
procedure reaches a minimum, the final atmospheric parameters are found.

The EW method
\citep[see e.g.][]{Sousa2008a,Mucciarelli2013,Bensby2014}, which we use
in this work, is based on the measurements of EWs from a list of lines
combined with the matching atomic data. The EW for a single line is
given as
\begin{align}
    \label{eq:EW}
    EW = \int_0^\infty \left(1 - \frac{F_\lambda}{F_0}\right) d\lambda,
\end{align}
where $F_0$ is the continuum level and $F_\lambda$ is the flux as a
function of wavelength.

Using this method, we obtain the abundance of individual lines by
the radiative transfer code MOOG \citep[][version 2013]{Sneden1973}
under the assumption of local thermodynamic equilibrium (LTE). To
obtain metallicity, we expect every spectral line of the same element
to produce the same abundance. In our analysis, we use neutral iron
(\ion{Fe}{i}) and single ionized iron (\ion{Fe}{ii}) as a proxy for
the metallicity. The effective temperature and surface gravity are
derived from the principles of ionization and excitation equilibrium
\citep[see][]{Gray2006}.


A disadvantage of the EW method may be a miscalculation of the EW. For
example, the placement of the continuum level may be incorrect, which
leads to an over- or underestimation of the EW for the given line.
Another source of error is contamination with either telluric lines or
other neighbouring lines. The relative error is typically larger for the
weak lines. In this work we will focus on the spectral region covered by
the Y, J, H, and K bands, which covers more than $\SI{15000}{\AA}$.



\subsection{Compiling the line list}

To compile the line list we used \emph{The Vienna Atomic Line Database} (VALD3) \citep{VALD1,VALD2}.
First, we downloaded a list of all iron lines present in the near
infrared region covering $10\,000\si{\AA}$ to $25\,000\si{\AA}$.
In total, $78\,537$ iron lines were found in this spectral region
($50\,198$ \ion{Fe}{i} lines and $28\,339$ \ion{Fe}{ii} lines).
Many of these lines are too faint to be detected in a spectrum
of a solar-type star. A spectrum of the Sun was downloaded from
the BASS2000 web page\footnote{The web page can be found here:
\url{bass2000.obspm.fr/solar_spect.php}} to select the best lines
for this analysis. The NIR part of the spectrum was obtained from
the Kitt Peak telescope \citep{Hinkle1995} at a resolution of
\SI{0.004}{\angstrom} at \SI{10000}{\angstrom} to \SI{0.1}{\angstrom}
at \SI{50000}{\angstrom}. The spectrum was downloaded in the highest
possible resolution at a given wavelength. The signal-to-noise ratio
of the spectrum varies from 3000 at $\SI{12000}{\AA}$ down to 1400 at
$\SI{21400}{\AA}$.

We use the \emph{Automatic Routine for line Equivalent widths in stellar Spectra} (ARES) software\footnote{The ARES software can be found
here: \url{http://www.astro.up.pt/~sousasag/ares/}. The following
settings were used: lambdai=7500, lambdaf=54000, smoothder=4, space=2.0,
rejt=0.995, lineresol=0.07, and miniline=2.}\citep{Sousa2007,Sousa2015a}
to automatically measure the EWs of all the lines. Since the first
version of ARES expects a 1D spectrum with equidistant wavelength
spacing, the solar spectrum was interpolated to a regular grid with
constant wavelength step of \SI{0.01}{\angstrom}. This did not change
the appearance of the spectrum, or therefore the EW. The EWs are
measured by fitting Gaussian profiles to spectral lines. For a given
line, ARES outputs the central wavelength of the line, the number of
lines fitted for the final result, the depth of the line, the FWHM of
the line, the EW of the line, and the Gaussian coefficients for the
line.

Once this step was done we then selected a subset of lines using the
following criteria:
\begin{itemize}
    \item If the number of fitted lines by ARES for a given line is higher than 10,
        this line is rejected because it is believed to be severely blended.
    \item If the EW is lower than \SI{5}{m\angstrom} for an absorption line, the strength
        is too low and it may be difficult to see the line in spectra with low
        signal-to-noise ratio or a spectrum with many spectral features.
    \item If the EW is higher than \SI{200}{m\angstrom} for a given line, the strength
        is too high and we can no longer fit the line with a Gaussian profile since
        the absorption line no longer has a pure Gaussian profile.
    \item If the fitted central wavelength is more than $\SI{0.05}{\AA}$
        from the wavelength provided by VALD3, the line will also be rejected to
        avoid false identification.
\end{itemize}
After the automatic removal of lines following the above criteria
we reduced the number of lines to 6060 and 2735 for \ion{Fe}{i} and
\ion{Fe}{ii}, respectively.



\subsection{Visual removal of lines}
\label{sub:visual_removal_of_lines}

A visual inspection of the lines is necessary at this point in order to
allow us to select only the best lines. We consider the best lines to be
the ones that are not blended, and therefore reliable EW measurements
can be made.

In this step we analyzed in detail small spectral windows
(\SI{3}{\angstrom} in width) around each line. For each spectral window,
the corresponding absorption lines for all elements were downloaded
from the VALD3 database. The location of these lines were plotted on
top of the solar spectrum, and any iron line was excluded if a line of
another element was present at the same wavelength. Iron lines were
also excluded when the absorption line was severely blended by other
spectral lines. Many of the removed iron lines at this step have higher
excitation potential than the final line list since these lines are
generally weaker than those with lower excitation potential. After this
step 593 \ion{Fe}{i} lines and 22 \ion{Fe}{ii} lines remained in the
sample.

For some spectral regions it was not clear which element or elements
caused an absorption line. In these cases the iron lines were marked for
further investigation with the synthesis explained below.


\subsection{Synthesis of selected lines}
\label{sub:synthesis_of_selected_lines}

Lines from all elements in a $\SI{6}{\AA}$ window around an iron line
marked for further investigation were used to make a synthetic spectrum.
The synthetic spectra were made with MOOG with the \emph{synth} driver.
We use an ATLAS9 atmosphere model \citep{Kurucz1993} with the nominal
solar atmospheric parameters $T_\mathrm{eff}=\SI{5777}{K}$, $\log
g = 4.438$, and $\xi_\mathrm{micro} = \SI{1.0}{km/s}$ to resemble
the Sun. We used three different iron abundances for the synthesis.
The first with solar iron abundance, the second with 0.2 dex above
solar, and the third with 0.2 dex below solar. We consider a solar iron
abundance of 7.47 as presented in \cite{Gonzalez2000}. This choice
of solar parameters and iron abundances was done to match the values
used by our team in previous papers \citep[see e.g.][and references
therein]{Santos13} and thereby to provide consistency within our group.
If the synthetic spectra shows variation at the absorption line of
interest with respect to the different iron abundances, then it is
likely to be an iron line. We also changed abundances of other elements
in the proximity to see if our line is blended with other elements. An
example of these plots can be seen in Fig.~\ref{fig:synthesis}.

\begin{figure}[tpb]
    \centering
    \includegraphics[width=1.0\linewidth]{figures/synthetic_spectrum.pdf}
    \caption{Top panel: Observed spectra in grey, while the
    coloured curves are synthetic spectra with increasing iron abundance
    as the two central lines get deeper. The iron abundance varies by a total
    of 0.4 dex. The vertical lines show all the places where there are
    iron lines in the line list. Bottom panel: The two curves show
    the difference between the first synthetic spectrum and the second
    ($\Delta_{21}$) and between the first synthetic
    spectrum and the third ($\Delta_{31}$); this highlights
    where the change in iron abundance has an impact.}
    \label{fig:synthesis}
\end{figure}

Sometimes more than one iron line might be present with very similar
wavelengths so they can no longer be resolved. In order to find the
iron line that is creating the observed absorption line, one of the two
were excluded from the line list for the synthetic spectra. If this
removed (either fully or partially) the absorption line in the synthetic
spectra, then it was considered the cause for the observed absorption
line, otherwise we excluded the line from the line list presented in
this work.

A few times two iron lines had identical wavelengths and excitation
potential. In those cases the $\log \mathit{gf}$ were combined (sum of
the $\mathit{gf}$-value) to create a single line that can be analyzed
with our method. We ended up with 414 and 12 lines of \ion{Fe}{i} and
\ion{Fe}{ii}, respectively.


\subsection{Calibrating the line list: astrophysical $\log$ gf values}
\label{ssub:Recalibrating-the-atomic-data}

The iron abundances for each line were calculated using the same
solar atmosphere model as described above for synthesis. This step
allowed us to remove possible outliers based on the assumption that
errors in the $\log \mathit{gf}$ values from the VALD3 database
would never lead to variations of the derived iron abundance of
more than 1 dex. All the \ion{Fe}{i} lines before recalibration
of the oscillator strength and removal of lines which deviates
more than 1 dex are presented in Fig.~\ref{fig:fe1_before_recal}.
We note that we only removed \ion{Fe}{i} lines here because the
\ion{Fe}{ii} lines were sparse and essential to determining the
surface gravity when we reach ionization balance, as explained in
Sect.~\ref{sec:deriving_parameters_with_the_ew_method}. After removal
of 1 dex outliers we were down to 319 and 12 lines, for \ion{Fe}{i} and
\ion{Fe}{ii} respectively.


After the removal of lines from the complete VALD3 line list we
recalibrated the oscillator strength of the lines ($\log\mathit{gf}$) in
order to match the adopted solar abundance, an inverse solar analysis.
This allowed us to perform a differential analysis other stars. Similar
approaches have been done by \citet{Sousa2008a,Onehag2012,Rhodin2015}.
In Fig.~\ref{fig:EWvsEP} the EWs of the iron lines present in the Sun
are plotted as a function of the excitation potential. This plot shows
the distribution after recalibration of $\log \mathit{gf}$ after the
cut for lines with abundances deviating more than 1 dex from the solar
value. The majority of the iron lines are found in H band as shown in
Fig~\ref{fig:solarspectrum}.



\begin{figure}[tpb]
    \centering
    \includegraphics[width=1.0\linewidth]{figures/EWvsEP.pdf}
    \caption{The distribution of \ion{Fe}{i} and \ion{Fe}{ii} lines
    in  blue and red, respectively. The distribution shows the
    measured EWs for the Sun as a function of the excitation potential.}
    \label{fig:EWvsEP}
\end{figure}


\begin{figure}[tpb]
    \centering
    \includegraphics[width=1.05\linewidth]{figures/solarspectrum.pdf}
    \caption{Distribution of both \ion{Fe}{i} and \ion{Fe}{ii} lines
    plotted on top of the solar spectrum. The distributions are for the final
    line list. There are two areas in the spectrum with high telluric
    contamination, which also mark the border between the filters we
    use from J to H around 14000\si{\angstrom} and from H to K around
    19000 \si{\angstrom}. Most of the lines are located in the H band.}
    \label{fig:solarspectrum}
\end{figure}


\subsection{Removal of high dispersion lines}
\label{sub:removal_of_unstable_lines}

To chose which line-derived abundances are less prone to errors caused
by the uncertainties in the EW measured, we decided to do the following
test. A Gaussian distribution was made for the EW of each line. We used
the width for the Gaussian distribution following the formula presented
in \citet{Caryel1988},
\begin{align}
    \sigma \simeq 1.6 \frac{\sqrt{\Delta\lambda\; \mathrm{EW}}}{\mathrm{S/N}},
\end{align}
where $\Delta\lambda=0.1\si{\angstrom}$ and we considered a
signal-to-noise ratio of 50, much lower than the signal-to-noise ratio
of the spectrum. This width was used to create a Gaussian distribution
with a mean around the original EW:
\begin{align}
    f(x, EW, \sigma) = \frac{1}{\sqrt{2\pi\sigma^2}} e^{-\frac{(x-EW)^2}{2\sigma^2}}.
\end{align}
We made 100 draws for each line and derived the abundance with solar
parameters, using the same atmospheric model as described above.
For each line we calculated the mean absolute deviation (MAD). The
MAD values are plotted against the original EWs in the upper part
of Fig.~\ref{fig:unstable_lines}. We see a clear trend towards
weaker lines, which is expected since a small absolute change in
the EW results in a large relative change in abundance, hence a high
MAD value. However, this does not mean that the abundances of these
lines have a high dispersion. Therefore, we detrended the data with
a fitted exponential. The residuals are shown in the lower part
of Fig.~\ref{fig:unstable_lines}. We use the detrended data as a
measurement for the dispersion of a given line. A single point above $3
\sigma$ is then removed iteratively until there are no longer any points
above this threshold. In this process we remove 33 lines. The final line
list is presented in Table~\ref{tab:linelist}.

\begin{figure}[tbp!]
    \centering
    \includegraphics[width=1.0\linewidth]{figures/unstable_lines.pdf}
    \caption{Upper plot: MAD values of 100 noisy lines
    with a simulated signal-to-noise ratio of 50. The red points are
    the 23 detected unstable lines, and the red curve is the last fit
    in the iterative removal of unstable lines. Lower plot: Detrended points from the upper plot, used for the $3\sigma$
    calculation for each iteration}
    \label{fig:unstable_lines}
\end{figure}



\begin{table}[tb!]
    \caption{The final line list as found for the Sun with astrophysical
    $\log\mathit{gf}$ values. A complete version of this table is  available
    online.}
    \label{tab:linelist}
    \centering
    \begin{tabular}{ccrrr}
      \hline\hline
        Wavelength [$\si{\angstrom}$] & Element      & EP [$\si{eV}$] & $\log\mathit{gf}$ &  EW [$\si{m\angstrom}$]   \\
      \hline
        10070.521                     & \ion{Fe}{i}  &     5.51       &      -1.527       &   6.6 \\
        10080.415                     & \ion{Fe}{i}  &     5.10       &      -2.008       &   5.3 \\
        10081.394                     & \ion{Fe}{i}  &     2.42       &      -4.551       &   6.2 \\
        10137.100                     & \ion{Fe}{i}  &     5.09       &      -1.768       &   9.2 \\
        10142.844                     & \ion{Fe}{i}  &     5.06       &      -1.574       &  14.4 \\
        10155.163                     & \ion{Fe}{i}  &     2.18       &      -4.340       &  15.8 \\
        10156.507                     & \ion{Fe}{i}  &     4.59       &      -2.125       &  11.8 \\
        10167.469                     & \ion{Fe}{i}  &     2.20       &      -4.199       &  19.8 \\
        10195.106                     & \ion{Fe}{i}  &     2.73       &      -3.625       &  21.9 \\
        10227.991                     & \ion{Fe}{i}  &     6.12       &      -0.449       &  19.4 \\
        10230.796                     & \ion{Fe}{i}  &     6.12       &      -0.414       &  21.0 \\
        10265.218                     & \ion{Fe}{i}  &     2.22       &      -4.668       &   7.7 \\
        10327.340                     & \ion{Fe}{i}  &     5.54       &       0.504       & 134.4 \\
        10332.328                     & \ion{Fe}{i}  &     3.63       &      -3.145       &  10.1 \\
        10340.886                     & \ion{Fe}{i}  &     2.20       &      -3.672       &  46.7 \\
        10347.966                     & \ion{Fe}{i}  &     5.39       &      -0.754       &  36.9 \\
        10353.805                     & \ion{Fe}{i}  &     5.39       &      -1.035       &  23.2 \\
        10364.063                     & \ion{Fe}{i}  &     5.45       &      -1.129       &  17.3 \\
        10379.000                     & \ion{Fe}{i}  &     2.22       &      -4.246       &  18.0 \\
        10388.746                     & \ion{Fe}{i}  &     5.45       &      -1.527       &   7.8 \\
          \ldots                      &   \ldots     &    \ldots      &      \ldots       &  \ldots\\
        10427.305                     & \ion{Fe}{ii} &     6.08       &      -1.662       &  12.4 \\
        10501.498                     & \ion{Fe}{ii} &     5.55       &      -1.926       &  18.2 \\
        10862.644                     & \ion{Fe}{ii} &     5.59       &      -2.043       &  14.8 \\
        11125.580                     & \ion{Fe}{ii} &     5.62       &      -2.301       &   9.1 \\
        11833.056                     & \ion{Fe}{ii} &     2.84       &      -3.379       &  81.7 \\
        12913.876                     & \ion{Fe}{ii} &     6.50       &       0.045       &  97.7 \\
        13251.144                     & \ion{Fe}{ii} &     9.41       &       0.860       &  13.0 \\
        13277.306                     & \ion{Fe}{ii} &     5.29       &      -2.043       &  35.9 \\
        13294.853                     & \ion{Fe}{ii} &     3.22       &      -3.613       &  56.9 \\
        13419.109                     & \ion{Fe}{ii} &     3.81       &      -3.484       &  32.8 \\
        15247.133                     & \ion{Fe}{ii} &     6.84       &      -1.691       &  10.5 \\
        15350.156                     & \ion{Fe}{ii} &     8.95       &       0.602       &  29.0 \\
        20460.070                     & \ion{Fe}{ii} &     1.67       &      -5.758       &  36.5 \\
      \hline
    \end{tabular}
\end{table}



\subsection{Deriving parameters with the EW method}
\label{sec:deriving_parameters_with_the_ew_method}

Once the EWs have been measured for all iron lines in the line list
(or for as many as possible), the next step is to derive the atmospheric
parameters. Atmosphere models are necessary for computing abundances of
the lines. The literature offers the possibility to choose from a wide
variety of model atmospheres. Models like ATLAS9 \citep{Kurucz1993} and
MARCS \citep{Gustafson2008} have been the preferred atmosphere models
for the derivation of spectroscopic parameters for FGK stars.

We use the ATLAS9 models which, for efficiency, are created in a grid
according to effective temperature, surface gravity, and metallicity.
In order to search for final parameters it is necessary to interpolate
models from the grid, thus allowing  a finer grid space
to be examined \citep[see e.g.][]{Sousa2014}. This grid of atmosphere models has been
used extensively by our group; it  allows us to work consistently over
multiple wavelength regions (optical and NIR).

For a given atmosphere model, abundances of all the lines in the line
list are calculated. By removing any correlation between the excitation
potential and abundance of all lines (from the same element) the effective
temperature is constrained. In a similar way, the microturbulence
can be constrained by removing any correlation between the reduced EW
($\log EW/\lambda$) and iron abundances, and the surface gravity is
found when there is ionization balance, i.e. the mean abundance of
\ion{Fe}{i} and \ion{Fe}{ii} are equal. Lastly, the iron abundance comes
from calculating the mean of all the iron abundances.
When there is no longer any correlation, the final atmospheric
parameters are obtained from the last atmosphere model.

In order to find the best atmosphere model, a minimization algorithm
is used based on the downhill simplex method \citep{Press1992}, which
searches  the parameter space for the best fitting atmospheric model,
i.e. the best parameters. The convergence criteria for the correlation
between excitation potential and abundances is a slope lower than
0.001. A slope lower than 0.002 for the correlation between the reduced
EW and the abundances, and a difference of less than 0.005 between
the mean abundances for \ion{Fe}{i} and \ion{Fe}{ii} is used in e.g.
\cite{Sousa2008a} and \cite{Tsantaki2013}.

The error estimate is based on the same method presented in
\citet{Gonzalez1998}. The uncertainty on $\xi_\mathrm{micro}$ is
determined from the standard deviation in the slope of abundance
versus reduced equivalent width; the uncertainty on $T_\mathrm{eff}$
is determined from the uncertainty on the slope of abundance
versus excitation potential in addition to the uncertainty on
$\xi_\mathrm{micro}$; the uncertainty on the iron abundance
is a combination of the uncertainties on $T_\mathrm{eff}$,
$\xi_\mathrm{micro}$, and the scatter of the individual \ion{Fe}{i}
abundances. The uncertainty on the surface gravity is based on the
uncertainty on $T_\mathrm{eff}$ and the scatter in \ion{Fe}{ii}
abundances.


\section{Results}
\label{sec:results}

\subsection{Derived parameters for the Sun}
\label{sec:derived_parameters_of_the_sun}

We derived the stellar atmospheric parameters for the Sun using the
resulting line list (including the solar calibrated astrophysical $\log
\mathit{gf}$ values). We used the minimization procedure described in
Sect.~\ref{sec:deriving_parameters_with_the_ew_method}. Since the line
list and $\log\mathit{gf}$ values were selected using the solar
spectrum, it is with no surprise that the derived parameters for the Sun
perfectly match the adopted solar values within the error bars as seen
in Table~\ref{tab:solar_params}.

Moreover, we derived parameters for different signal-to-noise ratios,
namely 25, 50, 100, 150, 225, and 300. The signal-to-noise ratios
were obtained by drawing EW from a Gaussian distribution with widths
dependent on the EW itself and the signal-to-noise as described above
in Sect.~\ref{sub:removal_of_unstable_lines}. For each considered
signal-to-noise, we made ten random line lists, giving us a total of 60
line lists. This exercise shows the expected precision for different
signal-to-noise ratios with the proposed line list. The final results
are presented in Fig.\ref{fig:snr_sun}. The error bars represent the
$3\sigma$ standard deviation from the ten different runs. As seen
from the figure, we expect to derive precise parameters (effective
temperature more precise than $\SI{50}{K}$, surface gravity with a
precision of 0.1 dex, iron abundance with a precision of 0.05, and
microturbulence with a precision of 0.3) down to a signal-to-noise ratio
of 50. At higher signal-to-noise ratios the precision increases. For example
at a signal-to-noise ratio of 100 the errors are reduced by a factor
of 2. The results can also be seen in Table~\ref{tab:solar_params}.
This shows that the line list is fully reliable for the whole range of
signal-to-noise ratios considered, even if the precision decreases at
lower signal-to-noise ratios as expected.

\begin{figure}[tbp!]
    \centering
    \includegraphics[width=1.0\linewidth]{figures/solar_parameters_snr.pdf}
    \caption{All plots show derived parameters as a function of the
    signal-to-noise. The error is the 3$\sigma$ standard deviation from
    the ten different runs for each signal-to-noise. The upper left plot
    shows the effective temperature. The upper right plot show the
    surface gravity ($\log g$). The lower left shows the iron abundance,
    used as a proxy for the metallicity. Finally, in the lower right
    plot the microturbulence is shown.}
    \label{fig:snr_sun}
\end{figure}

\begin{table*}[htb!]
    \caption{The derived parameters for the Sun at different
    signal-to-noise ratios. The error is the 3$\sigma$ standard
    deviation calculated from the ten runs at each signal-to-noise
    ratio.}
    \label{tab:solar_params}
    \centering
    \begin{tabular}{lllll}
      \hline\hline
        S/N & $T_\mathrm{eff}$ (K) & $\log g$ (dex)  &  [Fe/H] (dex)    & $\xi_\mathrm{micro}$ (km/s)  \\
      \hline
  Original  &  $5776 \pm 0$        & $4.43 \pm 0.00$ & $0.00 \pm 0.00$  & $0.99 \pm 0.00$              \\
      \hline
        25  &  $5808 \pm 119$      & $4.50 \pm 0.20$ & $0.00 \pm 0.06$  & $1.28 \pm 0.51$              \\
        50  &  $5780 \pm 41$       & $4.45 \pm 0.09$ & $0.00 \pm 0.02$  & $1.06 \pm 0.26$              \\
       100  &  $5776 \pm 22$       & $4.44 \pm 0.03$ & $0.00 \pm 0.01$  & $1.02 \pm 0.10$              \\
       150  &  $5776 \pm 12$       & $4.44 \pm 0.02$ & $0.00 \pm 0.00$  & $1.01 \pm 0.09$              \\
       225  &  $5779 \pm 10$       & $4.44 \pm 0.01$ & $0.00 \pm 0.00$  & $1.01 \pm 0.07$              \\
       300  &  $5777 \pm 10$       & $4.43 \pm 0.02$ & $0.00 \pm 0.00$  & $1.01 \pm 0.06$              \\
      \hline
    \end{tabular}
\end{table*}



\subsection{Derived parameters for HD20010}
\label{sec:derived_parameters_of_hd20010}

To test our new line list we search for a well-studied solar-type
star. The spectrum for such a target needs to be available in the NIR at
both high resolution and high signal-to-noise. An ideal place to look
for such a star is the CRIRES-POP database \citep{Lebzelter2012}. Here,
the best target for testing is HD20010, an F8 subgiant star. This star
has been part of many surveys and is therefore well studied. Different
parameters from the literature are listed in Table~\ref{tab:parameters}.

\begin{table*}[htb!]
    \caption{Selection of literature values for the atmospheric
    parameters for HD20010. The mean and a $3 \sigma$ standard
    deviation is presented at the end of the table from the literature
    values included, which we use as a reference for our
    derived parameters.}
    \label{tab:parameters}
    \centering
    \begin{tabular}{l|llll}
      \hline\hline
     Author                 & $T_\mathrm{eff}$ (K) & $\log g$ (dex)  &  [Fe/H] (dex)    & $\xi_\mathrm{micro}$ (km/s)  \\
      \hline
    \cite{Balachandran1990} & $6152$               & $4.15$          & $-0.27\pm0.08$   & $1.6$                        \\
    \cite{Favata1997}       & $6000$               & \ldots          & $-0.35\pm0.07$   & \ldots                       \\
    \cite{Santos2004}       & $6275\pm57$          & $4.40\pm0.37$   & $-0.19\pm0.06$   & $2.41\pm0.41$                \\
    \cite{Gonzalez2010}     & $6170\pm35$          & $3.93\pm0.02$   & $-0.206\pm0.025$ & $1.70\pm0.09$                \\
    \cite{Ramirez2012}      & $6073\pm78$          & $3.91\pm0.03$   & $-0.30\pm0.05$   & \ldots                       \\
    \cite{Mortier2013}      & $6114$               & \ldots          & $-0.19$          & \ldots                       \\
      \hline
      Mean                  & $6131\pm255$         & $4.01\pm0.60$   & $-0.23\pm0.14$   & $1.90\pm1.08$                \\
      \hline
    \end{tabular}
\end{table*}

The data available at CRIRES-POP are in the raw format and pipeline
reduced, while three small pieces of the spectra are fully reduced on the web
page\footnote{\url{http://www.univie.ac.at/crirespop/data.htm}}. The data
is in the standard CRIRES format with each fits file including four
binary tables with the data from the four detectors. In the future, the final
reduced data will be presented by the CRIRES-POP team. In contrast to the pipeline reduced data, this
will be of higher quality, a better wavelength calibration, and telluric correction. We
measured the EWs of the pipeline reduced spectra, and where there was
an overlap with the fully reduced spectrum, we measured both as a
consistency check. The measured EWs from the fully reduced spectra were
consistent with the measured EWs from the pipeline reduced spectra. As
mentioned above, we use the Y, J, H, and K bands which are all available
for this star. The spectra come in pieces of $\SI{50}{\angstrom}$
to $\SI{120}{\angstrom}$. These pieces  overlap each
other, and we were able to measure the EW for a single line up to
five times. Unfortunately, wavelength calibration is a difficult task
for CRIRES owing to the rather small spectral regions measured on each
detector. Each calibration was performed separately for each detector
and required the availability of a sufficient number of calibration
lines in the respective spectral region. This was not always the case
and a default linear solution was applied. A pipeline reduced spectrum
shows up as a stretched spectrum if the wavelength calibration is poor
compared to  a model spectrum or a solar spectrum, for example. The wavelength
calibration does not have any effect on the signal-to-noise ratio, which
is generally high for the spectrum of HD20010. The signal-to-noise
varies between 200 and 400 for different chunks. The pipeline reduced
spectra for HD20010 contains tellurics and the wavelength is shifted in
radial velocity. All of these factors make the line identification very
difficult, and so we developed a program\footnote{The program
(plot\textunderscore{}fits) is open source and can be found here:
\url{https://github.com/DanielAndreasen/astro_scripts}} to properly
identify the lines, which  does the following:

\begin{enumerate}
    \item Plotting the observed spectrum;
    \item Overplotting a model spectrum. In this particular case the solar spectrum was
        used since the atmospheric parameters are close enough, so the sun was able to
        serve as a model;
    \item Overplotting a telluric spectrum from the TAPAS web
          page\footnote{\url{http://ether.ipsl.jussieu.fr/tapas/}} \citep{Bertaux2014};
    \item Overplotting vertical lines at the location of lines in the list;
    \item Calculating the cross-correlation function (CCF) for the telluric spectrum
        with respect to the observed spectrum, locating the maximum value by a Gaussian fit,
        and using this to shift the telluric spectrum with the found RV;
    \item Performing the same as  step 5, but for the model;
    \item Shifting the lines with the same RV as found for the model/solar spectrum.
\end{enumerate}
The final plot shows the shifted spectra, and the CCFs at the sides. An
example of the software in use is shown in Fig.~\ref{fig:plot_fits}. The
two RVs are part of the title of the plot.

\begin{figure*}[tbp!]
    \centering
    \includegraphics[width=1.0\linewidth]{figures/plot_fits.pdf}
    \caption{The middle plot shows a piece of HD20010 (black), the model
    spectrum, in this case the Sun (green), a telluric spectrum (red),
    and two lines from our line list (magenta vertical lines). The
    plot to the left shows the CCF of the Sun with a fitted Gaussian.
    The right plot shows the same as the one to the left, but for the
    telluric spectrum.}
    \label{fig:plot_fits}
\end{figure*}

Once the lines were identified, the EWs were measured with the
\emph{splot} routine in \emph{Image Reduction and Analysis Facility} (IRAF). The reason not to choose ARES for this
task was to visually confirm the identification of the line given the
relative poor wavelength calibration. We were able to measure 249
\ion{Fe}{i} lines and 5 \ion{Fe}{ii} lines compared to 344 \ion{Fe}{i}
lines and 13 \ion{Fe}{ii} lines for the Sun over the whole NIR spectral
region. Whenever we had more than one measurement of a line, the average
was used for the final EW.

We derived the stellar parameters using the standard procedure (see
Sect.~\ref{sec:deriving_parameters_with_the_ew_method}) as done for
the Sun. Given the relatively low quality of the spectrum of HD20010
(see below) and because  it is not corrected for telluric
contamination, we made a cut in EW at 5\si{m\angstrom} in order to
remove the lines which are most affected by contamination from either
telluric or other line blends. Additionally, we made a cut in EP at
\SI{5.5}{eV}\footnote{We also derived the stellar parameters without
any cut in the EP, but the resulting values were always overestimated
(e.g. fixing $\log g$ to 4.01 we obtained a temperature of 6660K and
metallicity of +0.19 dex).} because the \ion{Fe}{i} and
\ion{Fe}{ii} lines usually used for stellar parameter determination
in the optical regime are also limited to similar values \citep[see
e.g.][]{Sousa2008a}. Higher excitation potential lines are also more
likely to be affected by non-LTE effects. When deriving the atmospheric
parameters, we made a $3\sigma$ outlier removal in the abundance
iteratively until there were no more outliers present. Since we could
only measure 5 \ion{Fe}{ii} lines, for comparison we also decided to
derive parameters using the same method, but we fixed the surface gravity
to the reference value. The resulting atmospheric parameters
and iron abundances are presented in Table~\ref{tab:hd20010}. The
effective temperature, surface gravity, and metallicity agree within the
errors with the literature values. Similar parameters are obtained by
fixing $\log g$ to the average literature value or by leaving it free.

\begin{table*}[htb!]
    \caption{The derived parameters for HD20010 with and without
    fixed surface gravity cut after 3$\sigma$ outlier removal.}
    \label{tab:hd20010}
    \centering
    \begin{tabular}{lllll}
      \hline\hline
                     & $T_\mathrm{eff}$ (K) &  $\log g$ (dex)  &   $\xi_\mathrm{micro}$ (km/s)  & [Fe/H] (dex)      \\
      \hline
        Literature   & $6131 \pm 255$       &  $4.01 \pm 0.60$ &    $1.90 \pm 1.08$              & $-0.23 \pm 0.14$ \\
      \hline
                     & $6116 \pm 224$       &  $4.21 \pm 0.58$ &    $2.45 \pm 0.45$              & $-0.14 \pm 0.14$ \\
                     & $6144 \pm 212$       &   4.01 (fixed)   &    $2.66 \pm 0.42$              & $-0.13 \pm 0.29$ \\
      \hline
    \end{tabular}
\end{table*}

The errors on the atmospheric parameters for HD20010 are much
higher than what is achievable with other measurements in the
literature, as presented above in Table~\ref{tab:parameters}. In order
to explain these errors, we calculated the abundances for all lines
which have at least two measurements of the EW. We then calculated the
abundances for the highest measured EW and the lowest. The differences
in abundances are presented in Fig.~\ref{fig:abundance_error}. The very
large differences (more than 0.1 dex) translate to the high errors in
the parameters.

\begin{figure}[tpb!]
    \centering
    \includegraphics[width=1.0\linewidth]{figures/abundance_error.pdf}
    \caption{Difference in abundance for lines in HD20010 that
    have at least two measurements of EW. The difference is calculated
    between the highest measured EW and the lowest. For the line list
    used for HD20010 we used the mean of all measurements available.}
    \label{fig:abundance_error}
\end{figure}

The source of the large errors on the parameters can be seen more
clearly where abundances are compared to excitation potential or
abundances versus reduced EW. Here the dispersion on the abundances can
be seen clearly, as shown in Fig.~\ref{fig:slopes}.

\begin{figure}[tpb!]
    \centering
    \includegraphics[width=1.0\linewidth]{figures/slopes.pdf}
    \caption{Top plot: \ion{Fe}{i} abundances for all
    lines are shown as a function of EP. Bottom plot: \ion{Fe}{i} abundances, but against the reduced EW. The high
    dispersion in the abundances leads to high error bars on the derived
    atmospheric parameters. The green lines are the slopes, and the
    dashed lines are the mean (under the green line), and the $3 \sigma$
    standard deviation.}
    \label{fig:slopes}
\end{figure}

This test strongly suggest that errors in the EWs, likely due to the
poor quality of this spectrum, are responsible for the relatively large
error bars in the derived stellar parameters. Systematic errors (e.g.
due to a possible non-optimal reduction of the spectrum) may be the
reason for these large error bars. As the CRIRES-POP team continue their
great efforts in reducing the optimal spectra, it will be interesting to
re-visit this star once the entire spectrum has been fully reduced.

\subsubsection{Surface gravity}
\label{subs:surface_gravity}
Although we have derived a consistent value for the surface gravity for
HD20010, given the small number of \ion{Fe}{ii} lines in the analysis,
we find this value to be of low precision and it should be considered with caution.
However, we emphasize that from our experience in using this method (the ionization
balance) in the optical, the other atmospheric parameters ($T_\mathrm{eff}$,
and [Fe/H]) have a low interdependency with the surface gravity. This has
been shown by \citet{Torres2012} and more recently by \citet{Mortier2014}.
Furthermore, with the upcoming results from the \emph{Gaia} mission we will
get precise surface gravity for a large number of stars and thus the best
option would be to fix this parameter if necessary.


\section{Conclusion}
\label{sec:conclusion}

In this work, we present a new iron line list for the NIR. The quality
of the line list plays a key role in deriving atmospheric stellar
parameters. Although the line list was compiled from a solar spectrum and
calibrated for the same, we tested it extensively for the slightly
hotter star HD20010. The first results with this line list are
promising. We also show that for a spectrum that contains telluric
lines, the best results appear when removing lines with an EW lower
than $\SI{5}{m\angstrom}$. In the future, the development of new high
resolution NIR spectrographs will allow us to obtain more high quality
spectra of stars in the whole FGK spectral range, thus allowing us to
better test and refine this line list.

Furthermore, it will be interesting to explore the use of this line list
to derive parameters for M-dwarf stars using high resolution and high
signal-to-noise NIR spectra. M-dwarf stars are especially interesting
targets for an exoplanetary viewpoint, since they are prone to forming
low-mass exoplanets \citep{Bonfils2013}. Hence, a precise analysis
of the host star's atmospheric parameters may greatly improve our
characterization of the possible exoplanets orbiting these low-mass
stars.

Lastly, with the upcoming NIR spectrographs  discussed above,
this work and future continuation will help the community to derive
atmospheric stellar parameters.



\begin{acknowledgements}

This work was supported by Funda\c{c}\~ao para a Ci\^encia e a
Tecnologia (FCT) through the research grants UID/FIS/04434/2013 and
PTDC/FIS-AST/1526/2014. N.C.S., and S.G.S. acknowledge the support from
FCT through Investigador FCT contracts of reference IF/00169/2012, and
IF/00028/2014, respectively, and POPH/FSE (EC) by FEDER funding through
the program “Programa Operacional de Factores de Competitividade
- COMPETE”. E.D.M. and B.J.A. acknowledge the support from FCT in
form of the fellowship SFRH/BPD/76606/2011 and SFRH/BPD/87776/2012,
respectively. This work also benefited from the collaboration of a
cooperation project FCT/CAPES - 2014/2015 (FCT Proc 4.4.1.00 CAPES).

This research has made use of the SIMBAD database operated at CDS,
Strasbourg (France).

This work has made use of the VALD database, operated at Uppsala
University, the Institute of Astronomy RAS in Moscow, and the University
of Vienna.

We thank the anonymous referee for useful comments and suggestions.
Finally, a special thanks to Dr. Thomas Lebzelter from the CRIRES-POP
team for helping with various questions on the spectra of HD20010.

\end{acknowledgements}


\bibpunct{(}{)}{;}{a}{}{,}
\bibliographystyle{aa}
\bibliography{thesis}


\begin{appendix}
\section{Iron abundances before recalibrating $\log$ gf}
\label{sec:section label}

It is clear from Fig.~\ref{fig:fe1_before_recal} that most of the lines taken
from VALD3 have bad $\log\mathit{gf}$ values. This reinforces our need to
use differential analysis also in the NIR.

\begin{figure*}[tbp!]
    \centering
    \includegraphics[width=1.0\linewidth]{figures/Fe1_before_recal.pdf}
    \caption{Abundances of all \ion{Fe}{i} lines before
    recalibration of the $\log\mathit{gf}$ values as a function of the wavelength.
    All red points deviate more than 1 dex from the expected solar value of
    7.47 (horizontal line) and are therefore discarded from the line list.}
    \label{fig:fe1_before_recal}
\end{figure*}


\end{appendix}




\end{document}
