\documentclass{aa}
\usepackage[varg]{txfonts}
\usepackage{siunitx}
\usepackage[version=3]{mhchem}


\def\eps{\varepsilon}
\def\aap{A\&A}
\def\apj{ApJ}
\def\apjs{ApJS}
\def\apjl{ApJL}
\def\mnras{MNRAS}
\def\aj{AJ}
\def\nat{Nature}
\def\aaps{A\&A Supp.}
\def\prd{Phys. Rev. D}
\def\prl{Phys. Rev. Lett.}
\def\araa{ARA\&A}       % Annual Review of Astron and Astrophys

\begin{document}


\title{NIR spectroscopy of the Sun and HD20010}
\subtitle{Compiling a new linelist in the NIR}

\subtitle{}

\author{D.~T.~Andreasen\inst{1,2,3}
    \and N.~C.~Santos.\inst{1,2}
    \and S.~G.~Sousa.\inst{1,2}
    \and E.~Delgado Mena.\inst{1,2}}


\institute{
Centro de Astrof\'isica, Universidade do Porto, Rua das Estrelas,
4150-762, Porto, Portugal
    \email{daniel.andreasen@astro.up.pt}
\and
Instituto de Astrof\'isica e Ci\^encias do Espa\c{c}o, Universidade do Porto, CAUP, Rua das
Estrelas, PT4150-762 Porto, Portugal
\and
Departamento de F\'isica e Astronom\'ia, Faculdade de Ci\^encias, Universidade do Porto, Portugal
}







\date{\today}

\abstract{}{}{}{}



\keywords{data reduction: high resolution spectra -- data reduction: low
    resolution spectra -- stars individual: HD20010 -- stars individual: Sun}
\maketitle



\section{Introduction}
\label{sec:introduction}
Effective surface temperature ($T_\mathrm{eff}$), surface gravity ($\log g$),
and metallicity ([Fe/H], where iron is used as a proxy) are fundamental
atmospheric parameters necessary to characterise a star, as well to determine
other inderict fundamental parameters, such as mass, radius, and age from
stellar evolutionary models \citep{Girardi2000}.

Precise and accurate stellar parameters are also essential in exoplanet
searches. Planetary radius, and mass are mainly found from lightcurve
analysis and radial velocity analysis, respectively. These parameters
are related to the stellar parameters.




% In this seminar I will explore the use of NIR spectroscopy towards
% the derivation of precise parameters for FGK dwarfs. This work will
% serve as a first approach to the determination of parameters in this
% spectral domain, using well known FGK dwarfs as benchmarks. NIR
% stellar spectra are significantly less blended, thus allowing for a
% precise line-by-line analysis \citep{Woolf2006,2012Onehag}. The NIR
% domain will then be key to solve the problem of M-dwarf parameter
% estimation. Note that the IR part of the spectrum is preferred over
% the optical for M-dwarfs because there is less photospheric molecular
% line blending and continuum depression.
%
% For the determination of stellar parameters we need to choose our
% approach. In general there is two widely used approaches to analyse
% spectra, 1: The spectral synthesis method, and 2: The equivalent
% method. I will use the latter approach.

\section{The method}
\label{sec:the_method}

Generally speaking there are two methods for determining parameters from
a spectrum. One is the spectral synthesis method, where a synthetic
spectrum is compared with an observed spectrum and by minimization
procedure the best fit is found between the synthetic spectrum and the
real spectrum \citep[see e.g.][]{2012Onehag}. The final atmospheric
parameters are found when the minimization procedure reach a minimum.

The other method is the equivalent width (EW) method, which we use in this
work.  With this method the EW are measured for all lines in a line list. The
EW is given as
\begin{align}
    \label{eq:EW}
    EW = \int_0^\infty \left(1 - \frac{F_\lambda}{F_0}\right) d\lambda,
\end{align}
where $F_0$ is the continuum level and $F_\lambda$ is the flux as a function
of wavelength. In other words, the EW is the area from a spectral line up
to the normalized continuum level.

With the EW method abundance for individual lines can be found with a code like
MOOG\footnote{The MOOG code can be downloaded free at
\url{http://www.as.utexas.edu/~chris/moog.html}}. By changing atmospheric
parameters as input for MOOG, similar abundances will be achieved for similar
elements when the best atmospheric parameters are chosen. Here we use neutral
iron and single ionized iron: FeI and FeII, respectively.

A disadvantage for this method, and a general problem with spectroscopy,
is the determination of the continuum flux level. Misplacement of the
continuum leads to wrong measurements of the EW. Many spectroscopic
features make it difficult to determine the continuum. This is
especially true for cool stars in the optical where molecular depression
and line blending is an issue. By moving the analysis to the NIR, we
reduce the molecular depression, and cooler stars such as M-dwarfs emit
more in this spectral region, why the continuum is at least easier to
determine.

We look at the spectral region covered by the J, H, and K bands, which cover a
spectral region larger than $\SI{15000}{\AA}$.



% The final stellar parameters are obtained when no correlation is present
% \citep{Gray2006}. The iron abundance is derived in this process, computed by
% the mean abundance given by all lines analysed.




\subsection{Compiling the line list}
To compile the line list we use the VALD
database\footnote{The VALD database can be found here:
\url{http://vald.astro.univie.ac.at/~vald3/php/vald.php}}. First we
download all iron lines present in the spectral region of interest. In
total 78537 lines of FeI and FeII are present in the spectral region
(FeI: 50198 and FeII: 28339). Many of these lines are to faint to see
in a spectrum. To select the lines for the line list a spectrum of the
Sun is used downloaded from the BASS2000 web page\footnote{The web
page can be found here: \url{bass2000.obspm.fr/solar_spect.php}}. The
spectrum were downloaded in the highest possible resolution. But since
the resolution is not constant the entire spectrum were interpolated to
a regular grid with constant wavelength step of $\SI{0.01}{\AA}$.

With the spectrum on a regular wavelength grid the
EW for all the lines were measured using the ARES
software\footnote{The ARES software can be found here:
\url{http://www.astro.up.pt/~sousasag/ares/}}\citep{Sousa2007}. This
software can measure EW automatically by fitting a Gaussian profile to
a spectral line. For a given line ARES output the central wavelength of
the line, the number of lines fitted for the end result, the depth of
the line, the FWHM of the line, the EW of the line, and last Gaussians
coefficients for the line.

We first remove lines from the line list based on the criteria listed
below:
\begin{itemize}
    \item If the number of fitted lines for a given line is higher than 10,
        this line is rejected because it is believed to be severely blended.
    \item If the EW is lower than $\SI{5}{m\AA}$ for a given line, the strength
        is too low and may be difficult to see in spectra with low S/N or a
        spectrum with many spectral features.
    \item If the EW is higher than $\SI{200}{m\AA}$ for a given line, the strength
        is too high and non-LTE effects are present in the core of the line.
    \item If the fitted central wavelength is more than $\SI{0.05}{\AA}$ away
        from the wavelength provided by VALD, the line will also be rejected to
        avoid false identification.
\end{itemize}




\subsection{Manually removal of lines}
\label{sub:manually_removal_of_lines}
After the automatic removal of lines following the criteria above
we reduced the number of lines to 6060 and 2735 for FeI and FeII
respectively. A manual inspection of the lines is necessary at this
point in order to sort bad lines. We only removed lines where we were
certain they did not belong to either FeI or FeII from our list.
The rest were kept as they would appear as outliers later on in the
analysis.

For the remaining lines, a small spectral window were created around them.
For each spectral range, the corresponding absorption lines for all elements
were downloaded. These lines were plotted on top of the Solar spectrum, and
iron lines were rejected when another element fitted the Solar absorption line
better.

For some spectral regions it was not clear which element or elements
caused an absorption line. In these cases the iron line were marked for further
investigation with synthesis.



\subsubsection{Synthesis of selected lines}
\label{sub:synthesis_of_selected_lines}
Lines from all elements in a small window around an iron line marked
for further investigation were used to make a synthetic spectrum.
The synthetic spectra were made with MOOG. We used 3 different iron
abundances for the synthesis. One with solar iron abundance, and
then two $\pm0.2$ dex. If the synthetic spectra shows variation at
the absorption line of interest with respect to the different iron
abundances, then it's likely to be an iron line. We also changed
abundances of other elements in the proximity to see if our line is
blended with other elements.

Sometimes more than 1 iron line might be present with very similar
wavelengths. In order to find the iron line which is creating the
absorption line, one of the two were removed from the line list for
the synthetic spectra. If this removed (either fully or partially) the
absorption line in the synthetic spectra, then it will be the cause for
the real absorption line.

A few times two iron lines had similar wavelengths and the same excitation
potential. In those cases the $\log \mathrm{gf}$ were combined.


\subsubsection{Recalibrating the atomic data}
\label{ssub:Recalibrating-the-atomic-data}

The iron abundances were calculated for all lines with an atmosphere
model characterised by $T_\mathrm{eff}=\SI{5777}{K}$, $\log g =
4.438$, $[Fe/H] = 0.0$, and $\xi_\mathrm{micro} = \SI{1.0}{km/s}$
to resemble the Sun. We consider a solar value of 7.47 according to
\cite{Gonzales2000}. Then, the lines showing an abundance that deviates
more than 1 dex were discarded. At this point we are down to 319 and 12
lines, for Fe I and Fe II respectively. We only removed Fe I lines here,
since the Fe II lines are sparse and important to determine the surface
gravity.

After the removal of lines from the complete VALD line list we only
need to recalibrate the strength of the lines ($\log \mathrm{gf}$)
the lines in order to match the adopted solar abundance. This is
a common thing to do for a star we know well \citep{2012Onehag}.
The abundance varies more or less linearly with $\log \mathrm{gf}$.
Therefore, the recalibrated $\log \mathrm{gf}$ can be found with few
iterations. In Figure~\ref{fig:Fe1_before_recal} the EW of the Fe I
lines are plotted with respect to the excitation potential. On top
of that, the color scale represent the abundance calculated before
the recalibration of $\log \mathrm{gf}$. A similar plot is seen in
Figure\ref{fig:Fe1_after_recal} but after the recalibration.

\begin{figure}[htpb]
    \centering
    \includegraphics[width=0.9\linewidth]{figures/EWvsEP.pdf}
    \caption{The distribution of Fe I lines with. At the x axis is the
    excitation potential, while the measure EW is shown at the y axis. The
    color scale indicates the abundance before recalibration.}
    \label{fig:Fe1_before_recal}
\end{figure}


\begin{figure}[htpb]
    \centering
    \includegraphics[width=0.9\linewidth]{figures/EWvsEP_cut.pdf}
    \caption{The distribution of Fe I lines. At the x axis is the
    excitation potential, while the measure EW is shown at the y axis.
    The histograms at the top and to the right of the plot, show the
    distribution of the respective axis.}
    \label{fig:Fe1_after_recal}
\end{figure}



\subsection{Deriving parameters with the EW method}
\label{sec:deriving_parameters_with_the_ew_method}

Once the EW's have been measured for all lines in the line list (or as
many as possible), the next step is to derive atmospheric parameters.
Atmosphere models are necessary for computing abundances of the lines.
The literature offers the possibility to choose from a wide variety
of model atmospheres. Models like ATLAS9 \citep{kurucz1993} and
MARCS \citep{gustafson2008} are the most used atmospheric models for
derivation of spectroscopic parameters for FGK stars.

Here we use the ATLAS9 models which, for efficiency, is created in
a grid according to effective temperature, surface gravity, and
metallicity. In order to search for final parameters it is necessary to
interpolate models from the grid, thus allowing to look in a finer grid
space.

For a given atmosphere model, abundances of all the lines in the line
list are calculated. By removing a correlation between the excitation
potential and abundance of all lines (from same element) the effective
temperature is constrained. In a similar way, the micro turbulence can
be constrained be removing any correlation between the reduced EW ($\log
EW$) and abundances. Lastly the surface gravity is found when there is
ionization balance, i.e. the mean abundance of FeI and FeII are equal.

When there are no longer any correlation, the final atmospheric
parameters are obtained from the atmospheric model to calculate the
given abundances. The iron abundance comes from calculating the mean of
all the iron abundances.

In order to find the best atmosphere model, a minimization algorithm
is used based on the downhill simplex method \citep{press1992} that
searches in the parameter space for the best fitting model. The
convergence criteria for the correlation between excitation potential
and abundances gives a slope less than 0.001, a slope less than 0.002
for the correlation between the reduced EW and the abundances, and a
difference of less than 0.005 between the mean abundances for FeI and
FeII.







\section{Results}
\label{sec:results}



\subsection{Derived parameters for the Sun}
\label{sec:derived_parameters_of_the_sun}

We derive atmospheric parameters for the Sun with the recalibrated line
list. This is a trivial case since the line list is calibrated for the
Sun, but it serves as a consistency test. Additionally, we added noise
to the EW measurements and derived parameters with different cuts in
excitation potential (EP). The noisy EW's is a drawn sample from a
Poisson distribution \begin{align} f(k; \lambda) = \frac{\lambda^k
e^{-\lambda}}{k!}, \end{align} for events with an expected separation
$\lambda$ the Poisson distribution $f(k; \lambda)$ describes the
probability of $k$ events occurring within the observed interval
$\lambda$. For all EW's we make 10 draws which corresponds to 11
different line lists (10 line lists with noisy EW's and the original
line list). We cut in EP since the absorption lines are generally weaker
for higher EP, thus the relative error in EW becomes higher. We want to
avoid this issue in other stars. The cuts are located at $\SI{5.5}{eV}$
and $\SI{5.0}{eV}$. This gives additionally 20 line lists for which we
derive atmospheric parameters.

In Figure~\ref{fig:solar_parameters} the parameters are plotted with
different EP cuts in the line list. The horizontal black line is the
derived atmospheric parameters without noise added to the EW. This agree
well with solar values (ADD REFERENCE HERE).

\begin{figure}[htpb]
    \centering
    \includegraphics[width=0.9\linewidth]{figures/solar_parameters_10runs.pdf}
    \caption{Derived atmospheric parameters for the Sun. The horizontal black
    lines are the parameters for the calibrated line list, the three points
    shows the median value of 10 runs with random noise added, the blue point
    is the result for the whole line list, and lastly the two green points are the
    results from the line lists cut in EP as given on the x-axis.}
    \label{fig:solar_parameters}
\end{figure}

By adding noise to the EW measurements and calculate the atmospheric
parameters with different cuts in EP, we see how we over estimate all
the parameters except the iron abundance, when a cut in EP is not
applied. This suggest that it might be necessary to do a cut in the line
list for other stars to get reliable parameters. This is indeed the case
for HD20010.



\subsection{Derived parameters for HD20010}
\label{sec:derived_parameters_of_hd20010}

HD20010 is a well studied F8 subgiant, see e.g.
\cite{Mortier2013,lebzelter2012}. Therefore, it is a prime object for
our studies, since we want to benchmark results from our new line list
with literature values. To analyse this star we used spectra from
the CRIRES-POP \citep{lebzelter2012}. The data comes in pieces of
$\SI{50}{\angstrom}$ to $\SI{120}{\angstrom}$. At the time of writing
the data is not yet fully reduced. This is a task the CRIRES-POP team
is working on. However, the data can still be used as it is after the
pipeline reduction. It is already known by the CRIRES-POP team that the
pipeline reduced form of the wavelength calibration is of poor quality.
Most of the spectra are stretched compared to e.g. a model or a solar
spectrum in the same region.

In order to measure the EW of the lines in our line list, the correct
absorption lines need to be identified. This was done by plotting
the spectra for HD20010, a solar spectrum (the parameters are close
enough so we can use the Sun as a reference), an observed telluric
spectrum from the TAPAS web page \citep{TAPAS}, and the lines from our
line list in the region we are looking at. The pieces of spectra are
shifted in radial velocity as well. A software was developed that plots
everything and calculate the cross correlation function of the model
(in this case the Sun), and the telluric spectra\footnote{The software
(plot\textunderscore{}fits) is open source and can be found here:
\url{https://github.com/DanielAndreasen/astro_scripts}}. The two CCFs
are fitted with a Gaussian, and the mid point is the RV. This two, in
principle, different radial velocities shift the model and the telluric
spectra. The lines from our line list are shifted with the same RV as for
the model. Figure~\ref{fig:plot_fits} shows an example of this.

\begin{figure}[htbp!]
    \centering
    \includegraphics[width=0.9\linewidth]{figures/plot_fits.pdf}
    \caption{The middle plot shows a piece of HD20010 (black), the model
    spectrum, in this case the Sun (green), a telluric spectrum (red), and two
    lines from our line list (magenta vertical lines). The plot to the left
    shows the CCF of the Sun with a fitted Gaussian. The right plot shows the
    same as the one to the left, but for the telluric spectrum.}
    \label{fig:plot_fits}
\end{figure}

The derived parameters for HD20010 for the full line list
are overestimated compared to literature values, see e.g.
\citet{Mortier2013,lebzelter2012}. To overcome this problem we
tried to remove lines above different EPs and re-determine the
atmospheric parameters. In this process we did not remove any of
the FeII lines (we were only able to measure the EW of 5 lines out
of the 13 from the solar case). Similar work as this has been done
in the visible range of the spectrum \citep{tsantaki2013}. In the
line list by \citep{tsantaki2013} the maximum EP is $\SI{5.1}{eV}$.
Figure~\ref{fig:HD20010_parameters_cuts} shows the results from this
exercise. In the minimization process the surface gravity were fixed
at 4.11(ADD REFERENCE) in order to reach convergence. This is a mean
value of previous studies. A second degree polynomial is fitted for the
effective temperature, iron abundance, and the micro turbulence with
the 95\% confidence interval. At a cut at $\SI{5.5}{eV}$ and below the
atmospheric parameters gets closer to the literature values which is
plotted as a horizontal black line.


\begin{figure}[htpb!]
    \centering
    \includegraphics[width=0.9\linewidth]{figures/HD20010_parameters_cuts.pdf}
    \caption{Atmospheric parameters for HD20010. In the top panel is the
    effective temperature. The middle panel is the iron abundance, and the
    bottom panel is the micro turbulence. The parameters are derived with
    different cuts in EP. Here the horizontal black lines are literature values.
    The parameters for the full line list are plotted to the right for all
    three plots.}
    \label{fig:HD20010_parameters_cuts}
\end{figure}











\newpage
\bibliography{thesis}
\bibliographystyle{astron}
\nocite*{}

\end{document}



















\section{Summary}
\label{sec:conclusion}
It is a lengthy process to compile a final good line list for spectral
analysis, and with a high possibility to make errors along the way. E.g.\ the
data that was first used from the \url{vso.nso.edu} was not corrected in radial
velocity was used. This was only discovered by an accident after many of the
above steps were already taken. Therefore, all the steps needed to be done
again, after switching to the BASS2000 dataset.

Now the solar spectrum are ready for analysis. All the neutral and first
ionized iron lines in a wide wavelength coverage in the NIR have been
downloaded from the VALD database. With the lines follows the excitation
potential and the oscillator strength. The latter has to be recalibrated for
all the lines before an analysis can take place.

Far from all the lines from the VALD database are useful for a spectral
analysis in the NIR domain. Therefore, many of the lines have been removed
following the selection criteria in Section~\ref{ssec:compiling_the_line_list}.

\subsubsection{The future work}
\label{ssub:the_future_work}
At the moment of writing the next step is to remove ``false-positive'' lines,
which are lines there are found in the VALD database and seems to hit a
spectral line, but in reality the spectral line belongs to another element.
This step has to be done before recalibrating the $\log gf$ values makes sense.
Even though yet another somewhat lengthy process awaits ahead, the
recalibration software is ready, as it was already shown in
Figure~\ref{fig:recalFe1}.

The recalibration will be done with respect to the Sun, the star we know best.
The goal is to characterise M-dwarfs, but since the Sun (G2V by definition,
\citet{Gray2006}) is not far away in spectral class, it should be save to use
the Sun as a starting point. However, for benchmarking the line list a sample
of FGK-dwarfs will be analysed. These stars will have well known stellar
parameters in order to test the line list. After the benchmarking some small
corrections might be made to the line list, in order to have consistent
results.

This will be the final step before the determination of stellar parameters
using iron lines in the NIR combined with the equivalent method on M-dwarfs,
and the characterisation of this cool planet hosts.


